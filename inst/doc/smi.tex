\documentclass[12pt]{article}

\title{Combining multiple imputations}
\author{Thomas Lumley}

%\VignetteIndexEntry{smi}{Combining multiple imputations}
\usepackage{/usr/local/thomas/Rarchive/r-devel/R/share/texmf/Sweave}
\begin{document}
\maketitle

Carlin \emph{et al.} (2003) illustrate the use of their Stata texttt for
multiple imputations with data from a cohort study of adolescent
health.  Five sets of imputations were done, separately for male and
female participants.  The resulting datasets are in \texttt{mitools/dta}.

First we read all the datasets into R, using \texttt{read.dta} from the \texttt{foreign} package.
\begin{Schunk}
\begin{Sinput}
> library(mitools)
> data.dir <- system.file("dta", package = "mitools")
> library(foreign)
> women <- imputationList(lapply(list.files(data.dir, pattern = "f.\\.dta", 
+     full = TRUE), read.dta, warn.missing.labels = FALSE))
> men <- imputationList(lapply(list.files(data.dir, pattern = "m.\\.dta", 
+     full = TRUE), read.dta, warn.missing.labels = FALSE))
\end{Sinput}
\end{Schunk}

We now combine the imputations for men and women, first defining a \texttt{sex} variable
\begin{Schunk}
\begin{Sinput}
> women <- update(women, sex = 0)
> men <- update(men, sex = 1)
> all <- rbind(women, men)
> all
\end{Sinput}
\begin{Soutput}
MI data with 5 datasets
Call: rbind(...)
\end{Soutput}
\begin{Sinput}
> colnames(all)
\end{Sinput}
\begin{Soutput}
 [1] "id"      "wave"    "mmetro"  "parsmk"  "drkfre"  "alcdos" 
 [7] "alcdhi"  "smk"     "cistot"  "mdrkfre" "sex"    
\end{Soutput}
\end{Schunk}

Now tabulate drinking frequency by sex
\begin{Schunk}
\begin{Sinput}
> with(all, table(sex, drkfre))
\end{Sinput}
\begin{Soutput}
[[1]]
   drkfre
sex Non drinker not in last wk <3 days last wk >=3 days last wk
  0 282         201            105              12             
  1 207         194            134              35             

[[2]]
   drkfre
sex Non drinker not in last wk <3 days last wk >=3 days last wk
  0 282         195            109              14             
  1 200         200            132              38             

[[3]]
   drkfre
sex Non drinker not in last wk <3 days last wk >=3 days last wk
  0 278         202            109              11             
  1 209         194            131              36             

[[4]]
   drkfre
sex Non drinker not in last wk <3 days last wk >=3 days last wk
  0 284         188            114              14             
  1 203         206            128              33             

[[5]]
   drkfre
sex Non drinker not in last wk <3 days last wk >=3 days last wk
  0 288         191            109              12             
  1 206         192            136              36             

attr(,"call")
with.imputationList(all, table(sex, drkfre))
\end{Soutput}
\end{Schunk}
and define a new `regular drinking' variables.
\begin{Schunk}
\begin{Sinput}
> all <- update(all, drkreg = as.numeric(drkfre) > 2)
> with(all, table(sex, drkreg))
\end{Sinput}
\begin{Soutput}
[[1]]
   drkreg
sex FALSE TRUE
  0 483   117 
  1 401   169 

[[2]]
   drkreg
sex FALSE TRUE
  0 477   123 
  1 400   170 

[[3]]
   drkreg
sex FALSE TRUE
  0 480   120 
  1 403   167 

[[4]]
   drkreg
sex FALSE TRUE
  0 472   128 
  1 409   161 

[[5]]
   drkreg
sex FALSE TRUE
  0 479   121 
  1 398   172 

attr(,"call")
with.imputationList(all, table(sex, drkreg))
\end{Soutput}
\end{Schunk}

We can now fit a logistic regression model for trends over time in drinking:
\begin{Schunk}
\begin{Sinput}
> model1 <- with(all, glm(drkreg ~ wave * sex, family = binomial()))
> MIcombine(model1)
\end{Sinput}
\begin{Soutput}
Multiple imputation results:
      with.imputationList(all, glm(drkreg ~ wave * sex, family = binomial()))
      MIcombine.default(model1)
                results         se
(Intercept) -2.25974358 0.26830731
wave         0.24055250 0.06587423
sex          0.64905222 0.34919264
wave:sex    -0.03725422 0.08609199
\end{Soutput}
\begin{Sinput}
> summary(MIcombine(model1))
\end{Sinput}
\begin{Soutput}
Multiple imputation results:
      with.imputationList(all, glm(drkreg ~ wave * sex, family = binomial()))
      MIcombine.default(model1)
                results         se      (lower     upper) missInfo
(Intercept) -2.25974358 0.26830731 -2.78584855 -1.7336386      4 %
wave         0.24055250 0.06587423  0.11092461  0.3701804     12 %
sex          0.64905222 0.34919264 -0.03537187  1.3334763      1 %
wave:sex    -0.03725422 0.08609199 -0.20623121  0.1317228      7 %
\end{Soutput}
\end{Schunk}

For model objects with \texttt{coef} and \texttt{vcov} methods the
extraction of coefficients and variances is automatic, but \texttt{MIextract} can still be used:
\begin{Schunk}
\begin{Sinput}
> beta <- MIextract(model1, fun = coef)
> vars <- MIextract(model1, fun = vcov)
> summary(MIcombine(beta, vars))
\end{Sinput}
\begin{Soutput}
Multiple imputation results:
      MIcombine.default(beta, vars)
                results         se      (lower     upper) missInfo
(Intercept) -2.25974358 0.26830731 -2.78584855 -1.7336386      4 %
wave         0.24055250 0.06587423  0.11092461  0.3701804     12 %
sex          0.64905222 0.34919264 -0.03537187  1.3334763      1 %
wave:sex    -0.03725422 0.08609199 -0.20623121  0.1317228      7 %
\end{Soutput}
\end{Schunk}


\subsection*{References}

Carlin JB, Li N, Greenwood P, Coffey C. (2003) Tools for analyzing multiply imputed datasets. \emph{Stata Journal} 3:1--20.

\end{document}
